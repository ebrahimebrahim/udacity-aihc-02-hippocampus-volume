\documentclass{article}
\usepackage[utf8]{inputenc}

\setlength\parskip{1em}
\setlength\parindent{0em}

\begin{document}

\section*{HippoVolume.AI - Validation Plan}

HippoVolume.AI is a piece of medical software centered on a deep neural network model that is trained
to segment MRIs for hippocampi.
The software uses the segmentation to
compute hippocampus volume and generates a report that can be integrated into a suitable clinician's workflow.

\subsection*{Intended Use}

The intended use of this software is to assist a radiologist in measuring hippocampus volume while reading a brain MRI.
It must be paired with out from the tool HippoCrop in order to function properly.

Indications for use:
One promising application of this software is in the diagnosis and progression-tracking of Alzheimer's disorder.
Hippocampus volume has been shown to be useful for this; see \cite{franko}, \cite{jack}, and \cite{moon}.

\subsection*{Training Data}

The training data was obtained from a repository at Vanderbilt University Medical Center.
Patients were recruited from the community in and surrounding Nashville, TN.
Some patients were healthy and some had a psychotic disorder, but all were free
from significant neurological illnesses or head injuries.
See \cite{data_paper} for more details.

Even though the training images were all obtained via the same medical imaging device (a Philips Achieva scanner),
we would like to claim that our software functions at acceptable levels on output from any brain MRI as long as there is sufficient contrast
to make out the relevant structures.

\textit{For later validating the model's performance, we will request more brain MRIs from a variety of devices.
We will request that the images be from patients who do not have any significant neurological illnesses or head injury,
and we will request that the contrast be sufficient to make out the hippocampi. We plan to use the HippoCrop tool to generate
input for our model.}

\subsection*{Ground Truth}
 
Ground truth for training was obtained by manual tracing of the posterior and anterior parts of the hippocampus.
The tracing was done following the protocol found in \cite{gt_protocol}.

\textit{
For later validating the model's performance, we will 
request ground truth labels manually traced by qualified specialists according to the protocol found in \cite{gt_protocol}.
}

\subsection*{Model Performance}

The current model's perforance was assessed using a Dice coefficient. We obtained a mean Dice coefficient of 0.9 on validation data
with the Dice coefficient not going below 0.8.

\textit{
For later validating the model's performance, we will
\begin{itemize}
\item apply HippoCrop to the MRI data obtained as described above,
\item generate predicted segmentations by applying our model,
\item and compute the Dice coefficient comparing the predicted segmentation to the ground truth obtained as described above.
\end{itemize}
If the mean dice coefficient is above 0.8 then the software can be considered to be performing acceptably.
}


\bibliographystyle{plain}
\bibliography{references}

\end{document}
